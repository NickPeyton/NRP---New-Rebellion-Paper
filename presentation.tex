\documentclass{beamer}

\usepackage{tabularx}
\usepackage{dcolumn}

\usetheme{Frankfurt}
\usecolortheme{beaver}
\useinnertheme{circles}

\setbeamercolor{itemize item}{fg=darkred}
\setbeamercolor{itemize subitem}{fg=darkred}
\setbeamercolor{itemize subsubitem}{fg=darkred}
\setbeamercolor{local structure}{fg=darkred}

\title{The Dissolution of the Monasteries and the Pilgrimage of Grace}
\author{Nick Peyton}
%\institute{London School of Economics}
\date{20 March, 2024}
\logo{\includegraphics[height=1cm]{Output/Images/LSE Logo.png}}
%%%%%%%%%%%%%%%%%%%%%%%%%%%%%%%%%%%%%%%%%%%%%%%%%%%%%%%%%%%%%%%%%%%%%
\begin{document}
    \frame{\titlepage}
%%%%%%%%%%%%%%%%%%%%%%%%%%%%%%%%%%%%%
    \section{Background}
    \begin{frame}
        \frametitle{Historical Background: The Dissolution of the Monasteries}
        \begin{columns}
            \column{.4\linewidth}
            \includegraphics[height=.99\textwidth, angle=270, origin=c]{Output/Images/dogFountains.jpg}
            \tiny Dirty animals now roam the halls of the once-mighty Fountains Abbey
            \column{.6\linewidth}
        \begin{itemize}
            \item Largest transfer of property in English history since 1066
            \item Began in 1536 with the dissolution of all religious houses with an annual income below \pounds200
            \item Large houses dissolved under pressure through 1540
            \item Immediately preceded by \emph{Valor Ecclesiasticus} of 1535, a full survey of all Church land in England
            \item Monasteries destroyed, land rapidly sold to raise money for war
        \end{itemize}
        \end{columns}
    \end{frame}

    
    \begin{frame}
        \frametitle{Historical Background: The Pilgrimage of Grace}
        \begin{columns}
            \column{.6\textwidth}
                \begin{itemize}
                    \item Largest rebellion between 1381 Peasants' Revolt and English Civil War
                    \item Rebellion sparked on October 1 by Dissolution Commissioners dissolving Louth Park Abbey in Lincolnshire
                    \item Spread rapidly throughout northern England and gathered an army of 30,000 rebels between October and December 1536
                    \item Confronted heavily outnumbered royal force near Doncaster, dispersed by false promises
                \end{itemize}
                
            \column{.4\textwidth}
            \includegraphics[width=.99\linewidth]{Output/Images/pogBanner.png}
            \tiny Banner of the Pilgrims, depicting the Five Wounds of Christ
        \end{columns}
    \end{frame}

    \begin{frame}
        \frametitle{Historical Background: The Pilgrimage of Grace}
    \begin{columns}
            \column{.6\textwidth}
            \begin{itemize}
                \item Wide-ranging list of demands:
                \begin{itemize}
                    \item Inheritance law
                    \item Taxation
                    \item Liturgy
                    \item Monasteries
                    \item Commoners on royal council
                \end{itemize}
                \item Historians generally agree that the Pilgrimage was driven by the commons
                \item Parish was the fundamental unit in raising and sustaining rebellion
                \item Most gentlemen involved claimed they were coerced, unclear how true this was
            \end{itemize}
            \column{.4\textwidth}
            \includegraphics[width=.99\linewidth]{Output/Images/pogBanner.png}
            \tiny Banner of the Pilgrims, depicting the Five Wounds of Christ
        \end{columns}
    \end{frame}
    
    \begin{frame}
        \frametitle{Literature}
        Three different literatures related to this topic:
        \begin{enumerate}
            \item Historical debate on causes of Pilgrimage
            \begin{itemize}
                \item Taxes and rumors (Hoyle 2001)
                \item Economic effects of Dissolution (Bush 1996)
                \item Broad sense of overturned divine order (Fletcher + MacCulloch 2008)
            \end{itemize}

            \item Literature on English rebellions
            \begin{itemize}
                \item "Village revolts" overwhelmingly motivated by agrarian issues (Manning 1988)
                \item Religion was the language of all popular revolt (Wood 2002)
            \end{itemize}
            
            \item Political science literature on causes of modern rebellions
            \begin{itemize}
                \item Game-theoretical models of revolt (Goldstone 1994, Granovetter 1978)
                \item Modernization and frustrated-desires view of revolt (Feierabend 1966)
                \item Deprivation theory of revolt (Kurer et al 2019)
            \end{itemize}
            
        \end{enumerate}
        
        
        
    \end{frame}
%%%%%%%%%%%%%%%%%%%%%%%%%%%%%%%%%
    \section{Data}
    \begin{frame}
        \frametitle{Core Data: The \emph{Valor Ecclesiasticus}}
        \begin{columns}
            \column{.55\textwidth}
            Survey of all Church property in England conducted in 1535, listed with annual income/expenditure value, typed up by Public Records Office in 19th century
                \begin{enumerate}
                    \item Source/destination, amount, and type of each entry
                    \item Georeference location using GENUKI, GBPN, KEPN, BHO
                    \item Join to spreadsheet of monasteries, then spatial join to parishes
                \end{enumerate}
            Final dataset contains roughly 12,000 entries for Northern counties
            \column{.45\textwidth}
            
            \includegraphics[width=.99\linewidth]{Output/Images/Healaugh Entry.png}
            \includegraphics[width=.99\linewidth]{Output/Images/Maps/HealaughStar.png}
            
        \end{columns}
    \end{frame}
    
    \begin{frame}
        \frametitle{Data: Land and Tithes}
        \begin{columns}
            \column{.5\textwidth}
            Land Owned by Monasteries
                \includegraphics[width=.9\linewidth]{Output/Images/Maps/llandOwned.png}
            \column{.5\textwidth}
            Tithes Collected by Monasteries
                \includegraphics[width=.9\linewidth]{Output/Images/Maps/lTitheOutT.png}
                
        \end{columns}
        \centering
        \hyperlink{limits}{\beamerskipbutton{Gap in Northeast}}
    \end{frame}


    
    \begin{frame}
        \frametitle{Rebellion Data}
        \begin{columns}
        
            \column{.55\linewidth}
            Drawn from \emph{The Pilgrimage of Grace: A Study of the Rebel Armies of October 1536} by ML Bush
            \begin{enumerate}
            \item Georeferenced by hand based on printed maps
            \item Pinpointed based on locations of nearby towns and rivers
            \item Spatially joined to parish shapefile
            \end{enumerate}

            \column{.45\linewidth}
                \includegraphics[width=.9\linewidth]{Output/Images/Maps/rebellion.png}
        \end{columns}
        
    \end{frame}
        \begin{frame}
        \frametitle{Rebellion Data Explained}
        \begin{columns}
        
            \column{.55\linewidth}

            \begin{itemize}
            \item \textbf{``Musters''} -- gatherings of rebels, initially repeated daily with rebels going home at night
            \item \textbf{``Primary''} musters -- subset of musters where people definitely gathered for the first time -- this is the \textbf{best indicator} of rebel sentiment
            \item \textbf{``Seats''} are the country seats of gentlemen voluntarily or involuntarily involved in the rebellion
            \end{itemize}


            \column{.45\linewidth}
                \includegraphics[width=.9\linewidth]{Output/Images/Maps/rebellion.png}
        \end{columns}
    \end{frame}
    \begin{frame}
        \frametitle{Supplementary Data}
        \begin{itemize}
            \item Monastery orders, locations, and dissolution dates from National Archives
            \item Lay Subsidy values digitized from maps in Sheail (1971)
            \item Terrain types from Atlas of Rural Settlement of England
            \item Slope and some lay subsidy variables from Heldring, Vollmer, and Robinson (2021)
        \end{itemize}
    \end{frame}
%%%%%%%%%%%%%%%%%%%%%%%%%%%%%%%%%
    \section{Methods and Results}
    \begin{frame}
        \frametitle{Methods: Parish-Level Regressions}
        \begin{math}
                    Rebellion_i = \beta_0 + \beta_1ln(Monastic Land_i) + \beta_2ln(Tithe_i) + \beta_3ln(Alms_i) + \beta_4Monastery_i + \beta_5ln(Taxes_i) + \beta_6TaxChangePct + \boldsymbol{Controls_i} + \epsilon
        \end{math}

        \bigskip
        \bigskip
        \begin{itemize}
            \item Rebellion = one of 3 rebellion variables: primary muster dummy, muster dummy, or number of seats of rebel gentlemen
            \item Monastery = presence of monastery or ln(monastic net income)
            \item Logistic regression used for musters, Poisson regression for rebel gentlemen
        \end{itemize}
    \end{frame}

    \begin{frame}
        \frametitle{Results: Parish Primary Muster Regression}
        
% Table created by stargazer v.5.2.3 by Marek Hlavac, Social Policy Institute. E-mail: marek.hlavac at gmail.com
% Date and time: Mon, Jun 03, 2024 - 3:51:04 PM
% Requires LaTeX packages: dcolumn 
\begin{table}[H] \centering 
  \caption{Primary Results: All Variables} 
  \label{tab:primary_all} 
\begin{tabular}{@{\extracolsep{.5pt}}lD{.}{.}{-3} D{.}{.}{-3} D{.}{.}{-3} D{.}{.}{-3} } 
\\[-1.8ex]\hline 
\hline \\[-1.8ex] 
 & \multicolumn{4}{c}{\textit{Dependent variable:}} \\ 
\cline{2-5} 
\\[-1.8ex] & \multicolumn{4}{c}{primary} \\ 
\\[-1.8ex] & \multicolumn{1}{c}{(1)} & \multicolumn{1}{c}{(2)} & \multicolumn{1}{c}{(3)} & \multicolumn{1}{c}{(4)}\\ 
\hline \\[-1.8ex] 
 ln(Land Owned) & 0.185^{***} & 0.182^{***} & 0.137^{**} & 0.135^{*} \\ 
  & (0.061) & (0.067) & (0.068) & (0.070) \\ 
  & & & & \\ 
 ln(Tithe) &  & -0.040 & -0.062 & -0.062 \\ 
  &  & (0.058) & (0.060) & (0.062) \\ 
  & & & & \\ 
 ln(Alms) &  & 0.246^{*} & 0.280^{*} & 0.271 \\ 
  &  & (0.138) & (0.153) & (0.174) \\ 
  & & & & \\ 
 ln(Net Income) &  & -0.105 & -0.139 & -0.219^{*} \\ 
  &  & (0.096) & (0.105) & (0.126) \\ 
  & & & & \\ 
 Friary &  & 0.939 & 0.454 & 0.343 \\ 
  &  & (0.650) & (0.709) & (0.727) \\ 
  & & & & \\ 
 ln(Lay Subsidy) &  &  & 0.098 & 0.085 \\ 
  &  &  & (0.193) & (0.214) \\ 
  & & & & \\ 
 Lay Subsidy Change &  &  & -1.966^{*} & -2.223^{*} \\ 
  &  &  & (1.159) & (1.334) \\ 
  & & & & \\ 
 ln(Population) &  &  & 0.250^{***} & 0.230^{***} \\ 
  &  &  & (0.055) & (0.065) \\ 
  & & & & \\ 
\hline \\[-1.8ex] 
Geographic Controls & N & N & N & Y \\ 
Observations & \multicolumn{1}{c}{1,033} & \multicolumn{1}{c}{1,033} & \multicolumn{1}{c}{1,033} & \multicolumn{1}{c}{989} \\ 
Log Likelihood & \multicolumn{1}{c}{-123.129} & \multicolumn{1}{c}{-120.723} & \multicolumn{1}{c}{-108.760} & \multicolumn{1}{c}{-104.192} \\ 
\hline 
\hline \\[-1.8ex] 
\textit{Note:}  & \multicolumn{4}{r}{$^{*}$p$<$0.1; $^{**}$p$<$0.05; $^{***}$p$<$0.01} \\ 
\end{tabular} 
\end{table} 

        \centering
        \hyperlink{conley}{\beamerskipbutton{Conley}}
        \hyperlink{kelly}{\beamerskipbutton{Kelly}}
    \end{frame}
    
    \begin{frame}
        \frametitle{Results: Parish Muster Regression}
        
% Table created by stargazer v.5.2.3 by Marek Hlavac, Social Policy Institute. E-mail: marek.hlavac at gmail.com
% Date and time: Mon, Jun 03, 2024 - 3:51:03 PM
% Requires LaTeX packages: dcolumn 
\begin{table}[H] \centering 
  \caption{Muster Results: All Variables} 
  \label{tab:muster_all} 
\begin{tabular}{@{\extracolsep{.5pt}}lD{.}{.}{-3} D{.}{.}{-3} D{.}{.}{-3} D{.}{.}{-3} } 
\\[-1.8ex]\hline 
\hline \\[-1.8ex] 
 & \multicolumn{4}{c}{\textit{Dependent variable:}} \\ 
\cline{2-5} 
\\[-1.8ex] & \multicolumn{4}{c}{muster} \\ 
\\[-1.8ex] & \multicolumn{1}{c}{(1)} & \multicolumn{1}{c}{(2)} & \multicolumn{1}{c}{(3)} & \multicolumn{1}{c}{(4)}\\ 
\hline \\[-1.8ex] 
 ln(Land Owned) & 0.133^{***} & 0.127^{***} & 0.087^{*} & 0.095^{*} \\ 
  & (0.042) & (0.047) & (0.048) & (0.050) \\ 
  & & & & \\ 
 ln(Tithe) &  & -0.064 & -0.080^{*} & -0.091^{*} \\ 
  &  & (0.047) & (0.048) & (0.052) \\ 
  & & & & \\ 
 ln(Alms) &  & 0.158^{*} & 0.174^{*} & 0.143 \\ 
  &  & (0.096) & (0.103) & (0.121) \\ 
  & & & & \\ 
 ln(Net Income) &  & -0.002 & -0.016 & -0.053 \\ 
  &  & (0.060) & (0.064) & (0.074) \\ 
  & & & & \\ 
 Friary &  & 1.007^{**} & 0.747 & 0.769 \\ 
  &  & (0.484) & (0.523) & (0.542) \\ 
  & & & & \\ 
 ln(Lay Subsidy) &  &  & 0.162 & 0.224 \\ 
  &  &  & (0.132) & (0.150) \\ 
  & & & & \\ 
 Lay Subsidy Change &  &  & -0.917 & -0.640 \\ 
  &  &  & (0.750) & (0.763) \\ 
  & & & & \\ 
 ln(Population) &  &  & 0.212^{***} & 0.171^{***} \\ 
  &  &  & (0.043) & (0.051) \\ 
  & & & & \\ 
\hline \\[-1.8ex] 
Geographic Controls & N & N & N & Y \\ 
Observations & \multicolumn{1}{c}{1,033} & \multicolumn{1}{c}{1,033} & \multicolumn{1}{c}{1,033} & \multicolumn{1}{c}{989} \\ 
Log Likelihood & \multicolumn{1}{c}{-197.649} & \multicolumn{1}{c}{-193.183} & \multicolumn{1}{c}{-181.014} & \multicolumn{1}{c}{-169.978} \\ 
\hline 
\hline \\[-1.8ex] 
\textit{Note:}  & \multicolumn{4}{r}{$^{*}$p$<$0.1; $^{**}$p$<$0.05; $^{***}$p$<$0.01} \\ 
\end{tabular} 
\end{table} 

    \end{frame}

    \begin{frame}
        \frametitle{Results: Parish Rebellious Gentlemen Regression}
        
% Table created by stargazer v.5.2.3 by Marek Hlavac, Social Policy Institute. E-mail: marek.hlavac at gmail.com
% Date and time: Mon, Feb 23, 2026 - 21:41:24
% Requires LaTeX packages: dcolumn 
\begin{table}[H] \centering 
  \caption{Seat Results: All Variables} 
  \label{tab:seat_all} 
\begin{tabular}{@{\extracolsep{.5pt}}lD{.}{.}{-3} D{.}{.}{-3} D{.}{.}{-3} D{.}{.}{-3} } 
\\[-1.8ex]\hline 
\hline \\[-1.8ex] 
 & \multicolumn{4}{c}{\textit{Dependent variable:}} \\ 
\cline{2-5} 
\\[-1.8ex] & \multicolumn{4}{c}{seats} \\ 
\\[-1.8ex] & \multicolumn{1}{c}{(1)} & \multicolumn{1}{c}{(2)} & \multicolumn{1}{c}{(3)} & \multicolumn{1}{c}{(4)}\\ 
\hline \\[-1.8ex] 
 ln(Land Owned) & 0.083^{***} & 0.072^{**} & 0.066^{*} & 0.078^{**} \\ 
  & (0.027) & (0.030) & (0.036) & (0.037) \\ 
  & & & & \\ 
 ln(Tithe) &  & 0.019 & 0.034 & 0.029 \\ 
  &  & (0.029) & (0.033) & (0.032) \\ 
  & & & & \\ 
 ln(Alms) &  & -0.025 & -0.008 & -0.028 \\ 
  &  & (0.069) & (0.072) & (0.074) \\ 
  & & & & \\ 
 ln(Net Income) &  & 0.021 & 0.009 & 0.008 \\ 
  &  & (0.033) & (0.036) & (0.036) \\ 
  & & & & \\ 
 Friary &  & -0.714 & -0.342 & -0.527 \\ 
  &  & (0.898) & (0.945) & (0.982) \\ 
  & & & & \\ 
 ln(Lay Subsidy) &  &  & 0.103 & 0.200^{**} \\ 
  &  &  & (0.080) & (0.086) \\ 
  & & & & \\ 
 Wet 1535 &  &  & 0.030 & 0.032 \\ 
  &  &  & (0.044) & (0.047) \\ 
  & & & & \\ 
 Wet 1536 &  &  & -0.078 & -0.078 \\ 
  &  &  & (0.145) & (0.150) \\ 
  & & & & \\ 
 Percy &  &  & 0.852^{***} & 0.592^{***} \\ 
  &  &  & (0.154) & (0.171) \\ 
  & & & & \\ 
 ln(Population) &  &  & 1.295^{***} & 1.126^{***} \\ 
  &  &  & (0.221) & (0.236) \\ 
  & & & & \\ 
\hline \\[-1.8ex] 
Geographic Controls & N & N & N & Y \\ 
Observations & \multicolumn{1}{c}{1,755} & \multicolumn{1}{c}{1,755} & \multicolumn{1}{c}{1,391} & \multicolumn{1}{c}{1,391} \\ 
Log Likelihood & \multicolumn{1}{c}{-429.207} & \multicolumn{1}{c}{-428.381} & \multicolumn{1}{c}{-303.332} & \multicolumn{1}{c}{-289.308} \\ 
\hline 
\hline \\[-1.8ex] 
\textit{Note:}  & \multicolumn{4}{r}{$^{*}$p$<$0.1; $^{**}$p$<$0.05; $^{***}$p$<$0.01} \\ 
\end{tabular} 
\end{table} 

    \end{frame}

   
    
    \begin{frame}
        \frametitle{Methods: Grid Regressions}
        \begin{columns}
            \column{.5\linewidth}
                \begin{itemize}
                    \item Same variables as ordinary parish regressions
                    \item Poisson regressions on 2, 5, 10, and 20-km square grid cells show similar effects to parishes
                    \item Results are robust to changes in parish boundaries
                    \item Results scale to larger areas
                \end{itemize}
            \column{.5\linewidth}
            \includegraphics[width=.8\textwidth]{Output/Images/Maps/grid10.png}
            \newline
            \tiny Monastic land, 10x10km grid cells
        \end{columns}
    \end{frame}

    \begin{frame}
        \frametitle{Results: Grid Regressions}
        
% Table created by stargazer v.5.2.3 by Marek Hlavac, Social Policy Institute. E-mail: marek.hlavac at gmail.com
% Date and time: Fri, Feb 20, 2026 - 16:58:01
% Requires LaTeX packages: dcolumn 
\begin{table}[H] \centering 
  \caption{Grid Regression Results} 
  \label{tab:grid} 
\begin{tabular}{@{\extracolsep{.5pt}}lD{.}{.}{-3} D{.}{.}{-3} D{.}{.}{-3} D{.}{.}{-3} } 
\\[-1.8ex]\hline 
\hline \\[-1.8ex] 
 & \multicolumn{4}{c}{\textit{Dependent variable:}} \\ 
\cline{2-5} 
\\[-1.8ex] & \multicolumn{4}{c}{muster} \\ 
\\[-1.8ex] & \multicolumn{1}{c}{(1)} & \multicolumn{1}{c}{(2)} & \multicolumn{1}{c}{(3)} & \multicolumn{1}{c}{(4)}\\ 
\hline \\[-1.8ex] 
 ln(Land Owned) & 0.229^{***} & 0.144^{***} & 0.179^{***} & 0.225^{*} \\ 
  & (0.043) & (0.045) & (0.065) & (0.124) \\ 
  & & & & \\ 
 ln(Tithe) & -0.062 & -0.008 & -0.028 & -0.021 \\ 
  & (0.058) & (0.041) & (0.037) & (0.044) \\ 
  & & & & \\ 
 ln(Alms) & 0.153 & 0.189^{**} & 0.037 & 0.033 \\ 
  & (0.095) & (0.080) & (0.054) & (0.045) \\ 
  & & & & \\ 
 ln(Net Income) & 0.028 & -0.058 & 0.050 & 0.038 \\ 
  & (0.058) & (0.049) & (0.037) & (0.040) \\ 
  & & & & \\ 
 Friary & 0.282 & 1.343^{***} & 1.042^{***} & 0.761^{***} \\ 
  & (0.601) & (0.410) & (0.346) & (0.275) \\ 
  & & & & \\ 
 ln(Lay Subsidy) & 0.123 & 0.220 & 0.404^{**} & 0.417^{**} \\ 
  & (0.143) & (0.148) & (0.174) & (0.184) \\ 
  & & & & \\ 
 Lay Subsidy Change & -2.434^{***} & -2.556^{***} & -3.300^{***} & -3.397^{***} \\ 
  & (0.898) & (0.968) & (1.168) & (1.276) \\ 
  & & & & \\ 
 ln(Population) & 0.289^{***} & 0.146^{***} & 0.038 & 0.095^{*} \\ 
  & (0.050) & (0.043) & (0.039) & (0.052) \\ 
  & & & & \\ 
 mean\_elev & -0.001 & -0.0003 & -0.001 & -0.0002 \\ 
  & (0.002) & (0.002) & (0.002) & (0.003) \\ 
  & & & & \\ 
\hline \\[-1.8ex] 
Geography & Y & Y & Y & Y \\ 
Observations & \multicolumn{1}{c}{10,823} & \multicolumn{1}{c}{1,845} & \multicolumn{1}{c}{513} & \multicolumn{1}{c}{160} \\ 
Log Likelihood & \multicolumn{1}{c}{-320.626} & \multicolumn{1}{c}{-242.463} & \multicolumn{1}{c}{-178.069} & \multicolumn{1}{c}{-139.429} \\ 
\hline 
\hline \\[-1.8ex] 
\textit{Note:}  & \multicolumn{4}{r}{$^{*}$p$<$0.1; $^{**}$p$<$0.05; $^{***}$p$<$0.01} \\ 
\end{tabular} 
\end{table} 

    \end{frame}

    \begin{frame}
        \frametitle{Results: Grid Regressions}
        
% Table created by stargazer v.5.2.3 by Marek Hlavac, Social Policy Institute. E-mail: marek.hlavac at gmail.com
% Date and time: Fri, Feb 20, 2026 - 16:58:02
% Requires LaTeX packages: dcolumn 
\begin{table}[H] \centering 
  \caption{Grid Regression Results} 
  \label{tab:grid} 
\begin{tabular}{@{\extracolsep{.5pt}}lD{.}{.}{-3} D{.}{.}{-3} D{.}{.}{-3} D{.}{.}{-3} } 
\\[-1.8ex]\hline 
\hline \\[-1.8ex] 
 & \multicolumn{4}{c}{\textit{Dependent variable:}} \\ 
\cline{2-5} 
\\[-1.8ex] & \multicolumn{4}{c}{seats} \\ 
\\[-1.8ex] & \multicolumn{1}{c}{(1)} & \multicolumn{1}{c}{(2)} & \multicolumn{1}{c}{(3)} & \multicolumn{1}{c}{(4)}\\ 
\hline \\[-1.8ex] 
 ln(Land Owned) & 0.097^{***} & 0.076^{***} & 0.210^{***} & 0.313^{***} \\ 
  & (0.033) & (0.029) & (0.040) & (0.088) \\ 
  & & & & \\ 
 ln(Tithe) & 0.062 & 0.039 & 0.011 & 0.084^{**} \\ 
  & (0.041) & (0.027) & (0.024) & (0.035) \\ 
  & & & & \\ 
 ln(Alms) & 0.055 & -0.008 & 0.050 & 0.032 \\ 
  & (0.109) & (0.063) & (0.043) & (0.036) \\ 
  & & & & \\ 
 ln(Net Income) & 0.004 & 0.014 & -0.043 & -0.025 \\ 
  & (0.067) & (0.036) & (0.028) & (0.026) \\ 
  & & & & \\ 
 Friary & -14.124 & -0.476 & -0.149 & -0.347 \\ 
  & (515.795) & (0.632) & (0.373) & (0.223) \\ 
  & & & & \\ 
 ln(Lay Subsidy) & 0.324^{***} & 0.440^{***} & 0.373^{***} & 0.314^{**} \\ 
  & (0.097) & (0.105) & (0.112) & (0.133) \\ 
  & & & & \\ 
 Lay Subsidy Change & -2.282^{***} & -3.641^{***} & -4.040^{***} & -4.298^{***} \\ 
  & (0.571) & (0.661) & (0.786) & (0.943) \\ 
  & & & & \\ 
 ln(Population) & 0.174^{***} & 0.077^{**} & -0.005 & 0.035 \\ 
  & (0.053) & (0.037) & (0.029) & (0.031) \\ 
  & & & & \\ 
 mean\_elev & 0.005^{***} & 0.004^{***} & 0.005^{***} & 0.007^{***} \\ 
  & (0.001) & (0.001) & (0.001) & (0.002) \\ 
  & & & & \\ 
\hline \\[-1.8ex] 
Geography & Y & Y & Y & Y \\ 
Observations & \multicolumn{1}{c}{10,823} & \multicolumn{1}{c}{1,845} & \multicolumn{1}{c}{513} & \multicolumn{1}{c}{160} \\ 
Log Likelihood & \multicolumn{1}{c}{-703.070} & \multicolumn{1}{c}{-471.239} & \multicolumn{1}{c}{-315.851} & \multicolumn{1}{c}{-194.158} \\ 
\hline 
\hline \\[-1.8ex] 
\textit{Note:}  & \multicolumn{4}{r}{$^{*}$p$<$0.1; $^{**}$p$<$0.05; $^{***}$p$<$0.01} \\ 
\end{tabular} 
\end{table} 

    \end{frame}
%%%%%%%%%%%%%%%%%%%%%%%%%%%%%%%%%%%%%%%%%%%%%%%%%%%%%%%%%
    \section{Discussion}
    \begin{frame}
        \frametitle{Discussion: Land}
        Why does monastic land predict rebellion?
        \begin{enumerate}
            \item Fear of changes in tenancy: inflation and rise of entry fines during 16th century
            \item Monastic recruiting shifted toward tenants and their neighbors in the century before 1536, monks did favors for family and friends
            \item Monasteries were a large and reliable source of demand for their tenants' agricultural products
            \item Median pound of land income for Northern monasteries came from less than 8km away
            \item Limited reports of monks pushing their tenants to rebel, arming them, etc.
        \end{enumerate}
        Could also be capturing some very rural population density not found in town and village population dataset
    \end{frame}

    \begin{frame}
        \frametitle{Discussion: Musters vs. Rebel Gentlemen}
        Results may indicate a difference in motivations between gentlemen and commons:
        \begin{itemize}
            \item Taxation level predicts rebel gentlemen in final specification
            \item Level of taxation does not provide information about the \emph{rate} of taxation
            \item Monastic land very weakly predicts rebel gentlemen, likely trades off directly with presence of \emph{any} gentleman's seat
            \item No map of non-rebel gentlemen, so high taxation per taxpayer may just reflect the presence of \emph{any} gentleman
        \end{itemize}
        
    \end{frame}
%%%%%%%%%%%%%%%%%%%%%%%%%%%%%%%%%%
    \section{Future Plans}
    \begin{frame}
        \frametitle{Future Plans}
        \begin{itemize}
            \item \textbf{Taxation:} Get estimates for total tax assessment of each hundred in 1334, valuation used until Tudor subsidies beginning in 1512
            \item \textbf{Names:} \emph{Valor} data contains names of monastic officials, initial results indicate $\sim$ 1/2 of rebel gentlemen were monastic officials
            \item \textbf{Case Study:} Find sample of monasteries with good cartulary records, analyze rebellion based on tenancy turnover
            \item \textbf{Heads of Houses:} The heads of many religious houses had been replaced by the Crown in the 10 years leading up to the Dissolution; may have affected tenants' propensity to rebel
            \item \textbf{Placebo:} Find another rebellion with granular data and not too much demand for more \emph{Valor} digitization
            
        \end{itemize}
    \end{frame}

    \begin{frame}{}
        \centering
        \Huge
        \textbf{Thank You!}
    \end{frame}
%%%%%%%%%%%%%%%%%%%%%%%%%%%%%%%%%%%%%%%%%%%%%%%%%%%%%%%%%%%%%%%%%%%%
    \appendix
    \section{Appendix: Robustness Checks}
    \begin{frame}[label=conley]
        \frametitle{Conley Standard Errors: Primary Muster Regression}
        \include{Output/Tables/conleyTable.tex}
    \end{frame}

    \begin{frame}[label=kelly]
        \frametitle{Kelly-esque Spatial Noise Test}
        Share of results of regressions on randomly-generated, spatially-autocorrelated primary muster patterns with p-values lower than main regression:
            \begin{itemize}
                \item Land: 13.6\%
                \item Tithe: 29.8\%
                \item Alms: 37.9\%
                \item Friary: 68.8\%
                \item Lay Subsidy: 68.5\%
            \end{itemize}
    \end{frame}

     \begin{frame}
        \frametitle{Methods: Inverse-Distance-Weighted Muster Regression}
        \begin{math}
                    Rebellion_i = \beta_0 + \beta_1ln(W_iMonastic Land_i) + \beta_2ln(W_iTithe_i) + \beta_3ln(W_iAlms_i) + \beta_4W_iMonastery_i + \beta_5ln(Taxes_i) + \beta_6TaxChangePct + \boldsymbol{Controls_i} + \epsilon
        \end{math}
        
        \bigskip
        \bigskip
        
        W is an inverse-distance-weighted spatial weights matrix with a maximum distance of 20km and a minimum distance of 1km (to avoid near-infinite values and divide-by-zeroes)
    \end{frame}
    \begin{frame}
        \frametitle{Results: Inverse-Distance-Weighted Parish Regression}
        \include{Output/Tables/idwMuster}
    \end{frame}

    \begin{frame}[label=limits]
    \frametitle{Limitations of the \emph{Valor}}
    \begin{columns}
        \column{.5\textwidth}
            \begin{itemize}
                \item Damage to original \emph{Valor} manuscript
                \item Daughter houses' property recorded with mother house
                \item Land often undervalued, rarely omitted
                \item Only perpetual alms from named benefactors listed, underestimating alms by at least 70\%
                \item ``Manors, land, tenants, and rent'' grouped together
            \end{itemize}
        \column{.5\textwidth}
        \includegraphics[width=.9\linewidth]{Output/Images/Maps/incomplete.png}
    \end{columns}
\end{frame}

\begin{frame}
    \frametitle{Accounting for Damage to the \emph{Valor}}
    \include{Output/Tables/regionSummary.tex}
\end{frame}
%%%%%%%%%%%%%%%%%%%%%%%%%%%%%%%%%%%%%%%%%%%%%%%%%%%%%%%%%%%%%%%%%%%%%
\end{document}